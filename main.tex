\documentclass{ctexart}
% Other classes are available too:
% \documentclass{ctexrep}
%\documentclass{ctexbook}
% \documentclass{ctexbeamer}

%% You can change the font if necessary.
% \setCJKmainfont{BabelStone Han}
% \setCJKsansfont{Noto Sans CJK SC}

\begin{document}
\title{星际争霸}
\author{dolphin} 
\maketitle

\tableofcontents

\begin{abstract}

\end{abstract}

\section{前言}
也许我们该从游戏中学一些东西,同时也能从工作与学习中找到乐趣。为自己的忍不住想玩找个借口,星际20年!虽然我生活在人类的星球上,但却向往星灵的魔法,渴望异虫的意识。毁灭为的是新生,战斗为的是自由。
\section{运筹学与策略学}
要隐藏自己的意图,学会长线发展。
均衡就是最佳对策
在对手所能允许的条件下,最大化自己的利益

\subsection{星际与其他策略游戏的对比与借鉴}
\subsubsection{桥牌}
对于实力的估计
\subsubsection{围棋}
相似性。从某种意义上,星际比的也是占地的多少,只不过除了很少数的平局,星际最后
放弃小的利益,争取大的利益
\subsubsection{象棋}
也许更像魔兽争霸

\section{虫族的策略}
\subsection{虫族的优点 }
\subsection{虫族缺点}
虫族的防御最贵
后期的对抗,大规模的兵团作战,不占优。
神族的闪电,海盗都是虫族的克星
人族的科技球,虫族基本无解。最多是打个平手。坦克海在陆地也很恐怖。
每次造建筑要损失农民
\subsection{论农民的数量}
一矿低配13农民,高配15农民,一矿应配两个基地,三矿应当满负荷配六基地,也许有更为精确的,但近似值是这样的。
\subsection{兵种的性价比}
狗子,性价比高,50矿/70血
刺蛇,性价比中,100元/80血
飞龙,性价比中,200元、120血,机动性
潜伏,性价比中,250元,125血
刺蛇性价比
\section{人族的策略}
\subsection{人族的优点和缺点}
人族的优点
可再生
很多强力兵种,火力强大
防御性强
反防御性强
人族的缺点
扩张性弱,对于操作要求高
生产建筑复杂

\end{document}
